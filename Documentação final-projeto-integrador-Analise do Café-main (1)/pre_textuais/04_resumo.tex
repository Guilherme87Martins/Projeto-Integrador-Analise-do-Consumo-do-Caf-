% !TEX root = ..\main.tex

\PalavrasChave{Consumo de café, Machine learning, Dados de compra, Padrão de consumo, Tipos de café, Hábitos por pessoa.}
\centeredchapterstyle
\begin{resumo}
    \noindent\textbf{Contexto}: O café é uma das bebidas mais populares do mundo, apreciada por milhões de pessoas por seu sabor, aroma e efeito estimulante. O consumo de café pode apresentar diversos benefícios à saúde, como redução do risco de doenças crônicas e melhora da função cognitiva. No entanto, o consumo excessivo pode levar a efeitos adversos, como insônia, ansiedade e problemas digestivos. \textbf{Objetivo}: Este estudo teve como objetivo analisar o padrão de consumo de café em uma população específica, utilizando dados coletados em um banco de dados que registra cada compra realizada. A análise exploratória visou identificar tendências, hábitos e características do consumo, a fim de fornecer informações relevantes para compreender melhor o comportamento dos consumidores e auxiliar na tomada de decisões relacionadas ao mercado de café.  \textbf{Método}: A pesquisa se baseou em uma análise exploratória de dados coletados em um banco de dados que registra cada compra de café realizada. Os dados foram organizados e processados para identificar variáveis relevantes, como tipo de café, quantidade comprada, frequência de compra e horário da compra. A seguir, foram aplicadas técnicas de estatística descritiva para descrever o perfil dos consumidores e caracterizar o padrão de consumo. \textbf{Resultados}: A análise exploratória revelou diversas informações relevantes sobre o consumo de café na população estudada. Entre os principais resultados, podemos destacar:
    
    -Tipos de café mais consumidos;
    
    -Quantidade média de café consumida por dia;
    
    - Frequência de compra;
    
    - Horários de maior consumo;
    
    - Perfil dos consumidores. \textbf{Conclusão}: A análise exploratória do consumo de café forneceu insights valiosos sobre o comportamento dos consumidores. Os resultados podem ser utilizados por empresas do setor cafeeiro para desenvolver estratégias de marketing mais direcionadas, lançar novos produtos e aprimorar a experiência do cliente. Além disso, a pesquisa contribui para uma melhor compreensão dos hábitos de consumo de café e seus possíveis impactos na saúde.
\end{resumo}
